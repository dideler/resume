%%%%%%%%%%%%%%%%%%%%%%%%%%%%%%%%%%%%%%%%%%%%%%%%%%%%%%%%%%%%%%%%%%%%%%
% My resume format is based on the LaTeX Template "Curriculum Vitae"
% by HowToTeX.com, though I've made some alterations to the style.
%%%%%%%%%%%%%%%%%%%%%%%%%%%%%%%%%%%%%%%%%%%%%%%%%%%%%%%%%%%%%%%%%%%%%%
\documentclass[paper=a4,fontsize=11pt]{scrartcl} % KOMA-article class

\usepackage[english]{babel}
\usepackage[utf8x]{inputenc}
\usepackage[protrusion=true,expansion=true]{microtype}
\usepackage[svgnames]{xcolor}  % Colours by their 'svgnames'
\usepackage{geometry}
  \textheight=700px  % Saving trees ;-)
\usepackage{url}

%% Define a new 'modern' style for the url package that will use a smaller font.
\makeatletter
\def\url@modernstyle{
  \@ifundefined{selectfont}{\def\UrlFont{\sf}}{\def\UrlFont{}}}
\makeatother
%% Now actually use the newly defined style.
\urlstyle{modern}

\frenchspacing              % Better looking spacings after periods
\pagestyle{empty}           % No pagenumbers/headers/footers

\renewcommand{\familydefault}{\sfdefault}

%%% Custom sectioning (sectsty package)
%%% ------------------------------------------------------------
\usepackage{sectsty}

\sectionfont{%                  % Change font of \section command
  \usefont{OT1}{phv}{b}{n}%   % bch-b-n: CharterBT-Bold font
  \sectionrule{0pt}{0pt}{-5pt}{3pt}}

%%% Macros
%%% ------------------------------------------------------------
\newlength{\spacebox}
\settowidth{\spacebox}{8888888888}      % Box to align text
\newcommand{\sepspace}{\vspace*{1em}}   % Vertical space macro

\newcommand{\MyName}[1]{ % Name
    \Huge \usefont{OT1}{phv}{b}{n} \hfill #1
    \par \normalsize \normalfont}

\newcommand{\MySlogan}[1]{ % Slogan (optional)
    \large \usefont{OT1}{phv}{m}{n}\hfill \textit{#1}
    \par \normalsize \normalfont}

\newcommand{\NewPart}[1]{\section*{\uppercase{#1}}}

\newcommand{\PersonalEntry}[2]{
    \noindent\hangindent=2em\hangafter=0 % Indentation
    \parbox{\spacebox}{                  % Box to align text
    \textit{#1}}                      % Entry name (birth, address, etc.)
    \hspace{1.5em} #2 \par}              % Entry value

\newcommand{\SkillsEntry}[2]{                % Same as \PersonalEntry
    \noindent\hangindent=2em\hangafter=0 % Indentation
    \parbox{\spacebox}{                  % Box to align text
    \textit{#1}}                    % Entry name (birth, address, etc.)
    \hspace{1.5em} #2 \par}              % Entry value

\newcommand{\AwardsEntry}[2]{                % Same as \PersonalEntry
    \noindent\hangindent=2em\hangafter=0 % Indentation
    \parbox{\spacebox}{                  % Box to align text
    \textit{#1}}                    % Entry name (birth, address, etc.)
    \hspace{1.5em} #2 \par}              % Entry value

\newcommand{\EducationEntry}[4]{
    \noindent \textbf{#1} \hfill      % Study
    \colorbox{Black}{
      \parbox{8.5em}{
      \hfill\color{White}#2}} \par  % Duration
    \noindent \textit{#3} \par        % School
    \noindent\hangindent=2em\hangafter=0 \small #4 % Description
    \normalsize \par}

\newcommand{\WorkEntry}[4]{       % Same as \EducationEntry
    \noindent \textbf{#1} \hfill      % Jobname
    \colorbox{Black}{%
      \parbox{9em}{%
      \hfill\color{White}#2}} \par   % Duration
        \noindent \textit{#3} \par        % Company
    \noindent\hangindent=2em\hangafter=0 \small #4 % Description
    \normalsize \par}

\newcommand{\OrganizationEntry}[4]{         % Similar to \EducationEntry
    \noindent \textbf{#1} \hfill            % Jobname
    \colorbox{Black}{\color{White}#2} \par  % Duration
    \noindent \textit{#3} \par              % Company
    \noindent\hangindent=2em\hangafter=0 \small #4 % Description
    \normalsize \par}

%%% Begin Document
%%% ------------------------------------------------------------
\begin{document}

\MyName{Dennis Ideler}
%\bigskip
%\MySlogan{To infinity and beyond...}
\bigskip
{\small dennisideler.com \quad github.com/dideler \quad linkedin.com/in/dennisideler \quad ideler.dennis@gmail.com}

\sepspace

%%% Personal details
%%% ------------------------------------------------------------
%\NewPart{Contact Info}{}
%
%\PersonalEntry{Website}{\url{dennisideler.com}}
%\PersonalEntry{LinkedIn}{\url{linkedin.com/in/dennisideler}}
%\PersonalEntry{GitHub}{\url{github.com/dideler}}
%\PersonalEntry{Email}{\url{ideler.dennis@gmail.com}}


%%% Work experience
%%% ------------------------------------------------------------
\NewPart{Work experience}{}

% NOTE:
% - Limit to top 3 jobs
% - Make the results measurable somehow (i.e. include numbers)
% - Preferably 1 page, do not exceed 2 pages!
% - Remove skills you're no longer familiar with
% - Show what you did, how you did it, and what the results were
%   e.g. "accomplished x by implementing y which led to z"
% - Try to keep info recent to ~5 years
% - Difficulty expressing accomplishments? Try the following words
%   Achieved, Improved, Trained/mentored, Managed, Created, Resolved,
%   Volunteered, Influenced, Increased/decreased, Ideas, Negotiated,
%   Launched, Revenue/profits, Under budget, Won

\WorkEntry{Software Engineer}{Feb 2015 - present}
{Funding Circle, London, UK}
{
 \begin{itemize} \itemsep -1pt
   \item Utilising agile and extreme programming techniques to continuously deliver reliable software,
   with a focus on compliance and making the engineering team more effective.
 \end{itemize}
 \textbf{Tools:} Ruby, Rails, RSpec, Cucumber, PostgreSQL, Git, GitHub, Capistrano, Jenkins, JIRA
}
\sepspace

\WorkEntry{Consultant (Remote)}{Sep 2014 - Feb 2015}
{Landlord Web Solutions, Ontario, Canada}
{
 \begin{itemize} \itemsep -1pt
   \item Continued development of metrics system for apartment industry
   % Working on lead tracking phase (future phases are metrics, then analytics)
 \end{itemize}
 \textbf{Tools:} Ruby, Rails, RSpec, PostgreSQL, Git, GitHub, Heroku, Mailgun, Python
}
\sepspace

\WorkEntry{Software Developer}{Jan 2014 - Jun 2014}
{Landlord Web Solutions, Ontario, Canada}
{
 \begin{itemize} \itemsep -1pt
   \item Improved effectiveness of marketing by revamping email communications
   % - researched how to improve open rates and design good emails
   %   - do a/b testing when unsure
   %   - send as a person, not a business (sender name, wording, etc)
   %   - KIS
   %   - limit the amount of graphics
   %   - focus on one thing, obvious (and multiple) CTAs for the same action
   %   - short non-spammy subject
   %   - optimize for mobile (easy to read, short, responsive, focused)
   %   - transactional emails have highest click-through rate,
   %     newsletters have somewhat low CTR,
   %     promotional emails have very low CTR
   %   - space out emails; don't send too many emails too often
   %   - send early in the morning
   %   - send to "organic" contacts who opted-in
   % - reviewed marketing emails before they were sent out
   % - started contributing to reallygoodemails.com/about in my free time
   % - increase delivery
   %   - researched how to reduce spam filter blockage
   %   - our e-blasts were failing spam tests (e.g. mail-tester.com)
   %   - do not cold email (especially with Canada's new anti-spam legislation)
   %   - segment email lists so emails are highly targeted (i.e. relevant content)
   %   - reduce bounce rates (which can hurt senders) by cleaning mailing lists
   %   - easy unsubscribe (ideally one click)
   %   - address in footer
   %   - don't use misleading/deceptive info
   %     (e.g. links that redirect or point elsewhere, false info, etc.)
   %   - use a reputable email service provider (ESP)
   %   - add alt tags to images
   % - designed and sent with constant contact
   % Also see https://gist.github.com/dideler/534e5996bbbbfe9e0f59
   \item Increased usability of products through UX feedback and user testing
   % - Collected usage videos via http://peek.usertesting.com/ and wrote a report
   % - Gave UX feedback and suggestions during hands-on meetings
   \item Trained team on development practices by sharing knowledge and leading by example
   % - Advocated and/or wrote blog posts about git workflows (e.g. tagging,
   %   branches), github workflows (e.g. using issues, releases, PRs),
   %   code review, testing, user testing, etc.
   % - Helped coworkers with git/github questions
   % - Wrote and updated documentation for some technical tasks
   \item Created a CRUD application for managing important aspects of our annual Landlord WEBCON conference, such as ticket sales, invoices, promo codes, and sponsors. Metrics gave key insights for guiding decision making, especially around measuring and boosting ticket sales.
   % Gain valuable knowledge from metrics.
   % Key insights that help drive the evolution of the conference.
   % Used to recommend action or to guide decision making rooted in business context.
   % Track ticket sales and vacancies and measure our advertising ROI
   % Measure and boost our impact on Twitter.
   % Dashboard for an overview and quickly measuring engagement
   \item Designed and began implementation of a metrics system for the apartment industry
 \end{itemize}
 \textbf{Tools:} Ruby, Rails, MySQL, CoffeeScript, Bootstrap, Git, GitHub, Heroku, Sendgrid
}
\sepspace

\WorkEntry{Systems Analyst (Co-op)}{May 2010 - Dec 2010}
{Ministry of Transportation of Ontario, Ontario, Canada}
{
 \begin{itemize} \itemsep -1pt
   \item Effectively collaborated with supervisors, co-workers, and
   consultants to migrate the old ministry-wide intranet site to a new
   SharePoint site that we designed, built, and maintained.
   \item Contributed to 6+ major projects over 8 months.
   Tasks ranged from interviewing project managers, to code refactoring,
   to business architecture and writing UML documentation.
   %• MTO Intranet Migration Project
   %• Training Management System
   %• Major Applications Portfolio Strategy
   %• eLearning Project
   %• Security eForm
   %• Green IT Initiative
 \end{itemize}
 \textbf{Tools:} Microsoft Sharepoint, JavaScript, jQuery, HTML, CSS, UML
} % Detailed MTO work experience: http://goo.gl/IQ2Uc
\sepspace

\WorkEntry{Data Specialist (Co-op)}{Jan 2009 - Apr 2009}
{GenieKnows Inc., Nova Scotia, Canada}
{
 \begin{itemize} \itemsep -1pt
   \item Enhanced business data for results used to provide the search
   and mapping features of GenieKnows Local Search (now yellowee.com)
   through Automated Data Collection.
   \item Researched specialized crawling strategies and presented findings
   to R\&D team. Utilizing the knowledge, built specialized web crawlers
   for sites on the Surface Web and Deep Web.
   %\item % Daily duties included: identifying data requirements, researching data sources, automating the downloading and processing/formatting of data, evaluating data quality, writing technical documentation, logging bugs, and logging progress.
 \end{itemize}
 \textbf{Tools:} Linux shell scripting (bash), Ruby, Watir, Selenium, Trac, Subversion (SVN)
}

%%% Organizations
%%% ------------------------------------------------------------
\NewPart{Organizations}{}

% TODO: Add OpenHack

\OrganizationEntry{Team Member}{2012 - 2014}
{Software Niagara, \url{softwareniagara.com}}
{
 \begin{itemize} \itemsep -1pt
   \item Grassroots organization which aims to improve the Niagara Region tech scene
   \item Organize social and informative events, such as DemoCamps, DevTricks, and Coworking
   \item Built \url{voteinpublic.com} with the team at a hackathon and demoed it at a DemoCamp
 \end{itemize}
}
\sepspace

\OrganizationEntry{Contributor \& Mentor}{2011 - present}
{Freeseer, \url{freeseer.github.io}}
{
 \begin{itemize} \itemsep -1pt % Reduce space between items.
   \item Free and open source screencasting software designed for conferences
   \item Write and maintain Freeseer's comprehensive documentation
   \item Collaborate using agile and open source processes in a distributed team
   \item Mentor student interns from Google Summer of Code (GSoC) and Undergraduate Capstone Open Source Programs (UCOSP)
 \end{itemize}
}
\sepspace

\OrganizationEntry{Founder \& Moderator}{2011 - present}
{Brock University Subreddit, \url{brocku.reddit.com}}
{
 \begin{itemize} \itemsep -1pt
   \item Online discussion board for topics related to BrockU
   \item Increased user engagement by contributing to the board and organizing meetups
   \item Executed marketing campaigns that grew the board to 300+ members
 \end{itemize}
}
\sepspace

\OrganizationEntry{Executive}{2009 - 2013}
{Computer Science Club, Brock University, \url{github.com/brockcsc}}
{
 \begin{itemize} \itemsep -1pt
   \item Created the Brock CSC organization on GitHub which has 5+ repositories and 10+ members
   \item Helped organize and market social and academic events
   \item Prepared ACM ICPC teams through tryouts and coaching (\url{github.com/brockcsc/acm-icpc})
 \end{itemize}
}

%%% Education
%%% ------------------------------------------------------------
\NewPart{Education}{}

\EducationEntry{B.Sc. Computer Science (Honours \& Co-op)}
{Sep 2007 - Oct 2013}
{Brock University, Ontario, Canada}
{
\begin{itemize}  \itemsep -1pt % Reduce space between items.
  \item Concentration in Intelligent Systems (Artificial Intelligence)
  \item Tutored introductory Computer Science, Math, and Applied Computing courses, 2011-2013
  \item Judged the Niagara Region Secondary Schools Programming Contest, 2011-2012
  \item Representative at Ontario Universities' Fair, Fall Preview Day,
  and CSEdWeek, 2009, 2012
  \item Guest speaker at Co-op Student Panel Workshop, 2010-2012
  \item Co-founded Chess Club and the BrockBots computer science
  message board, 2009-2010
  \item Deputy leader of a 7-person team that produced a cross-platform media center, 2009-2010
\end{itemize}
}

%%% Projects (TODO)
%%% ------------------------------------------------------------
%Maestro
%-------
%what: built a powerful cross-platform media center with 6 teammates, that's
%      capable of playing music, viewing photos, and playing videos, and can
%      be easily extended via it's plugin system design
%how: using c++ and Qt (also Phonon, a cross-platform multimedia framework
%     that enables the use of audio and video content in Qt applications
%results: highest graded project in course, fully functional, and met deadlines
%measurable: over the course of two semesters (one semester for planning and design, the other for
%            implementation)

%%% Skills
%%% ------------------------------------------------------------
\NewPart{Skills}{}

%% When running low on space, consider using {\small text goes here} for skills
\SkillsEntry{Software}{Ruby, Ruby on Rails, Python, C++, Java, JavaScript, Git, GitHub,}
\SkillsEntry{}{Shell Scripting (Bash, Fish), Vim, Linux, Android, SQL, \LaTeX, IRC}
\sepspace
\SkillsEntry{Languages}{Dutch (intermediate)}

%%% Awards -- quick hack, should be improved in macro section
%%% ------------------------------------------------------------
\NewPart{Awards}{}

\AwardsEntry{Competition}{$\bullet$ Facebook Hacker Cup, Online Round 1 Qualifier, 2013}
\AwardsEntry{}{$\bullet$ ACM International Collegiate Programming Contest (ACM ICPC)}
\AwardsEntry{}{\hspace{5pt} Honorable Mention, 2008-2012}
\AwardsEntry{}{$\bullet$ CS Games, Top 10, 2010-2011}
\AwardsEntry{}{$\bullet$ Great Canadian Appathon, Top 25, 2010}
\sepspace
\AwardsEntry{Academic}{$\bullet$ Deans' Honours List, Brock University, 2011}
\AwardsEntry{}{$\bullet$ Residence Action Council Member of the Year,}
\AwardsEntry{}{\hspace{5pt} Quarry View Residence, Brock University, 2008}

%%% References
%%% ------------------------------------------------------------
%\NewPart{References}{}
%Available upon request
\end{document}
